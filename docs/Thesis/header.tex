\documentclass[11pt, a4paper]{report}
\usepackage{polski}
\usepackage[cp1250]{inputenc}
\usepackage{color}
\usepackage[table]{xcolor}
\usepackage{listings}
\usepackage{caption}
\usepackage{enumitem}
\usepackage{graphics}
\usepackage{epsfig}
\usepackage{rotating}
\usepackage{fancyhdr}
\usepackage{subfig}
\usepackage{url}
\usepackage{float}

\usepackage{booktabs}
\newcommand{\ra}[1]{\renewcommand{\arraystretch}{#1}}

\renewcommand{\abstractname}{Abstract}

\restylefloat{figure} 

\renewcommand{\figurename}{Rys.}
\renewcommand{\tablename}{Tab.}
\renewcommand{\lstlistlistingname}{Spis listing�w}


%% Define a new 'leo' style for the package that will use a smaller font.
\makeatletter
\def\url@leostyle{%
  \@ifundefined{selectfont}{\def\UrlFont{\sf}}{\def\UrlFont{\small\ttfamily}}}
\makeatother
%% Now actually use the newly defined style.
\urlstyle{leo}

\pagestyle{fancy}


\pagestyle{fancy}
\rhead{}
\lhead{\nouppercase{\leftmark}}



\lstdefinelanguage{apache}{
morekeywords={ServerAdmin, ServerName, ServerAlias, DocumentRoot, Options, AllowOverride, Order, ErrorLog, LogLevel, CustomLog, allow},
sensitive=false,
morestring=[b]"
}

\definecolor{bbb}{gray}{0.90}

\newcommand{\lstMikro}[1]
{\lstinline[breakatwhitespace=true, postbreak=\kern-3ex, basicstyle=\small\ttfamily,breaklines=true,language=bash,literate={\`}{}{0\discretionary{-}{}{}}]$#1$}

\lstset{
         basicstyle=\scriptsize\ttfamily, % Standardschrift
			numbers=left,
         numberstyle=\tiny,          % Stil der Zeilennummern
         %stepnumber=2,               % Abstand zwischen den Zeilennummern
         numbersep=5pt,              % Abstand der Nummern zum Text
         tabsize=2,                  % Groesse von Tabs
         extendedchars=true,         %
         breaklines=true,            % Zeilen werden Umgebrochen
         keywordstyle=\color{blue},
         stringstyle=\color{red}\ttfamily, % Farbe der String
         xleftmargin=17pt,
         framexleftmargin=17pt,
         framexrightmargin=6pt,
			framextopmargin=0pt,
         backgroundcolor=\color{bbb},
         showstringspaces=false     % Leerzeichen in Strings anzeigen ?        
 }

\floatstyle{ruled}
\newfloat{program}{thp}{lop}
\floatname{program}{Listing}

\DeclareCaptionFont{white}{\color{white}}
\DeclareCaptionFormat{listing}{\colorbox{gray}{\parbox{\textwidth}{#1#2#3}}}
\captionsetup[lstlisting]{format=listing,labelfont=white,textfont=white}

\makeatletter
\newcommand{\linia}{\rule{\linewidth}{0.4mm}}

% new itemize environment with itemsep parameter
\newenvironment{packed_item}[1][0]
  { \begin{itemize}
    % set spacing between items
    \addtolength{\itemsep}{#1\baselineskip}
    % set spacing between lines
    \addtolength{\baselineskip}{#1\baselineskip} }
  { \end{itemize} }


\renewcommand{\maketitle}{\begin{titlepage}

    \vspace*{1cm}

    \begin{center}\small
    Politechnika ��dzka\\
    Wydzia� Fizyki Technicznej i Matematyki Stosowanej\\
    Informatyka\\
    Systemy Informatyczne i Bazy Danych
    \end{center}

    \vspace{3cm}

    \noindent\linia

    \begin{center}

      \LARGE \textsc{\@title}

         \end{center}

     \linia

    \vspace{0.5cm}

    \begin{flushright}

    \textit{\small Autor:}\\

    \normalsize \textsc{\@author} \par

    \vspace{1cm}

     \textit{\small Opiekun:}\\

         dr in�. Anety Poniszewska - Mara�da
		
     \end{flushright}

    \vspace*{\stretch{6}}

    \begin{center}

    \@date

    \end{center}

  \end{titlepage}%

}

\makeatother
